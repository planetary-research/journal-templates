% Planetary Research Open Diamond Journal 
% Author Submission Template 
% Version: 1.2 – 27 Nov 2025
\documentclass{planetary}

% -------------------- Title block --------------------
% - Fill in title
\title{<Article Title>}
% -------------------- Author block --------------------
% - Fill in authors, ORCID, and affiliations below.
\author[1,*]{First Author\,\orcidlink{0000-0000-0000-0000}}
\author[2]{Second Author\,\orcidlink{0000-0000-0000-0000}}
\affil[1]{Affiliation of First Author, City, Country, ROR ID}
\affil[2]{Affiliation of Second Author, City, Country, ROR ID}
\affil[*]{\textit{Corresponding author:} First Author — \texttt{first.author@university.educ}}

% -------------------- BIB FILE --------------------
\addbibresource{references.bib} % replace by the name of your .bib file
% -------------------- Date --------------------
\date{\small Submission date: \today}

% -------------------- Document --------------------
\begin{document}


% --------------------------------------------------

\maketitle
\thispagestyle{firstpage}

% -------------------- Manuscript type -------------
% The manuscript type must be one of: 
%     Article, Review, Letter, Numerical code, Dataset,
%     Missions and instrumentation, Introduction,
%     Commentary, or Editorial.
\mstype{Article}

% -------------------- Abstract --------------------
\begin{abstract}
The abstract must be between 250 and 350 words, without line breaks. It should provide a clear understanding of the work without requiring the reader to consult the full article. The abstract must be self-contained: it should not include references, and any abbreviations or acronyms must be introduced. The abstract should provide the context or background for the study and should state the study's purpose, basic procedures, main findings, and principal conclusions. Authors should ensure that it accurately reflects the content of the article.
\end{abstract}

\vfill 
\noindent
\begin{tabular}{p{0.01\textwidth} p{0.32\textwidth} p{0.01\textwidth} p{0.43\textwidth} p{0.1\textwidth}}
\cline{1-5}
\scriptsize{\textcopyright} 
& \scriptsize{The Author(s) 2026.} 
& 
& \multirow{2}{*}{\parbox{0.43\textwidth}{\vspace{1mm}\scriptsize{This work is licensed under a \href{https://creativecommons.org/licenses/by/4.0/}{Creative Commons Attribution License 4.0} which permits unrestricted use, distribution, and reproduction in any medium, provided the original work is properly cited.}}} 
& \vspace{1mm}\includegraphics[height=1.5em]{cc.png}\hspace{0.15em}\includegraphics[height=1.5em]{by.png} 
\end{tabular}




\newpage
% -------------------- Main Sections --------------------
\section{Introduction}

All elements required for compilation must be supplied. Compilation should be error-free.

Please refrain from using any self-made definitions ("newcommand") since these will get lost during further conversion of your text. If you use typing abbreviations, search and replace them before submitting your article.

\subsection{Subsection}

Sections and subsections should be numbered, with a maximum of 3 levels (1.; 1.1.; 1.1.1.).

% --------------------- Figure example ---------------------------
\begin{figure}[!ht]
    \centering
    \includegraphics[width=0.3\textwidth]{planet.png}
    \caption{All lettering (symbols, numbers) in figures must be consistent in style and size across all illustrations and large enough to ensure legibility. Each figure and table must be accompanied by a concise, explanatory caption. Captions must include any necessary credits. Each figure must correspond to a single, separate file: the use of commands such as subfigure or equivalent packages is not allowed; only one "includegraphics" command per figure is permitted. See the figures section in the authors submission guidelines, \href{https://planetary-research.org/submission-guidelines}{https://planetary-research.org/submission-guidelines}}
    \label{fig:enter-label}
\end{figure}

\subsubsection{Sub-subsection}

Authors should make the best possible use of cross-references, both when calling up tables and figures and when calling up references.

References should be cited in sentences: as indicated in~\textcite{becker2015phd}, \textcite{Rudolph2020}, \textcite{bagnasco2017nirspec}, and \textcite{Breuer2015}.

References can also be cited between brackets~\parencites{dressler2006nasa,Neukumetal2001a,Wieczorek2022,SJOGRENdataset}.

All figures and tables should be cited in the text (see Table~\ref{tab:example_dummy} and Fig.~\ref{fig:enter-label}).

Numbered equations can also be cited (see Eq.~\eqref{eq:loiHooke}).

% ---------------------- Equation example -----------------------

\begin{equation}
     \sigma^{ij}({\bf x},t)
     \;=\;
     \sigma_0^{ij}({\bf x},t)
     \;+\;
     c_0^{ijkl}({\bf x})
     \,
     \varepsilon^{kl}({\bf x},t)\ ,
     \label{eq:loiHooke}
\end{equation}
where the term  
$ \sigma_0^{ij}({\bf x},t) $  
is called the \emph{pre-stress}, and the coefficients  
$ c_0^{ijkl}({\bf x}) $  
are the \emph{elastic constants}.  

 %---------------------Table example ---------------------
\begin{table}[!ht]
    \centering
    \renewcommand{\arraystretch}{1.0}
    \footnotesize{
        \begin{tabular}{c c}
            \toprule
            \textbf{Name} & \textbf{Variable} \\
            \hline
            Dummy parameter A before event & $\alpha_{\text{before}}$ \\
            Dummy parameter B before event & $\beta_{\text{before}}$ \\
            Sum of Gaussian differences in A & $\Sigma\text{DoG}(\Delta\alpha)$ \\
            \textit{Normalized Dummy Index} & NDI \\
            \hline
        \end{tabular}}
        \caption{Example table with placeholder content. Replace labels and symbols as needed.}
        \label{tab:example_dummy}
\end{table}

\textbf{Acknowledgement}. Specify contributions that require acknowledgement but do not warrant a place among the authors, acknowledgements for technical assistance, material or financial support. When persons are mentioned in the acknowledgements, the corresponding author must have requested their permission to appear in the text.

\section*{Statements}

\textbf{Author contribution.} See the related information in the policies section of the website, \href{https://planetary-research.org/authorship}{https://planetary-research.org/authorship}

\textbf{Code availability.} See the related information in the open science section of the website, \href{https://planetary-research.org/open-science}{https://planetary-research.org/open-science}
The code availability statement should include the following information:
\begin{itemize}
\item Whether any code was used or developed as part of the study, and for what purpose (e.g., data preprocessing, modeling, visualization, statistical analysis, etc.).
\item Whether the code is publicly available, and if so:
        \begin{itemize}
			\item The name of the repository where the code is hosted.
			\item The exact URL or DOI providing direct access to the code.
			\item The license under which the code is released (e.g., MIT, GPLv3, Apache 2.0).
			\item The version or commit hash used for reproducibility.
			\item Any installation or usage instructions, or a reference to a README file.
			\end{itemize}

\item If the code is not publicly available, a justification must be provided (e.g., proprietary software, third-party restrictions, security concerns).
\item If only part of the code is shared, this should be clearly stated, including which components are available and which are not.
\item If no code was used, this should also be clearly stated.
\end{itemize}


\textbf{Data availability.} See the related information in the open science section of the website, \href{https://planetary-research.org/open-science}{https://planetary-research.org/open-science}
Articles which include results based on research data must include a statement with the following information:
\begin{itemize}
\item A list of results based on the data.
\item A clear specification of the nature of the data, its format, the method of collection, and the method of analysis.
\item The location where the data are deposited: name of the repository, dataset title, version used, a persistent identifier or URL (e.g., DOI), and the license under which the dataset is available.
\item If a subset of the data is used, it must be clearly identified.
\item If the data are under embargo, the reason must be specified (e.g., legal, ethical, or commercial restrictions).
\item If no data are associated with the work, this must be explicitly stated.
\end{itemize}

\textbf{Funding.} All sources of funding for the work must be disclosed.
For work funded by a research agency, the following information must be provided:
\begin{itemize}
\item Name of the funding agency
\item Name of the funding program
\item Title of the funded project
\item Grant number or funding reference
\end{itemize}

\textbf{Links of interest.} The journal follows international practice regarding links of interest. All manuscript submissions must be accompanied by a declaration of interest. All authors should therefore declare any links of interest that may arise from their work in general and that may interfere with the analysis, presentation of data or results. It is recommended to list both financial and non-financial interests: past or present employment, consulting activity, financing, intellectual property, ownership of financial assets, advisory roles, participation in a lobby, personal relationship…
The declaration of interest related to the article should be listed as follows:
\begin{itemize}
\item In case there are no links of interest with the authors of the submitted article, the following statement should be added directly in the manuscript: The author(s) declare(s) that he (they) has (have) no relationship of interest.
\item If one or more of the authors of the article has (have) a relationship of interest, the complete list of these authors must be mentioned.
The initials of the author(s) concerned, and the name of the associated company should be added to the exhaustive list of potential links of interest.
\end{itemize}

\textbf{Use of AI.} See the related information in the policies section of the website, \href{https://planetary-research.org/use-of-artificial-intelligence}{https://planetary-research.org/use-of-artificial-intelligence}




% -------------------- Appendix --------------------
\appendix
\section{Appendix Title}

% -------------------- Bibliography --------------------
\printbibliography

% -------------------- License & copyright --------------------
\vfill 

\noindent
\begin{tabular}{p{0.01\textwidth} p{0.32\textwidth} p{0.01\textwidth} p{0.43\textwidth} p{0.1\textwidth}}
\cline{1-5}
\scriptsize{\textcopyright} 
& \scriptsize{The Author(s) 2026.} 
& 
& \multirow{2}{*}{\parbox{0.43\textwidth}{\vspace{1mm}\scriptsize{This work is licensed under a \href{https://creativecommons.org/licenses/by/4.0/}{Creative Commons Attribution License 4.0} which permits unrestricted use, distribution, and reproduction in any medium, provided the original work is properly cited.}}} 
& \vspace{1mm}\includegraphics[height=1.5em]{cc.png}\hspace{0.15em}\includegraphics[height=1.5em]{by.png} 
\end{tabular}

\end{document}